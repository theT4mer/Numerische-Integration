% Header: Here are all packages used and some additional definitions
%%%%%%%%%%%%%%%%%%%%%%%%%%%%%%%%%%%%%%%%%%%%%%%%%%%%%%%%%%%%%%%%%%%

\documentclass[11pt,a4paper]{scrartcl}
\usepackage[margin=2.5cm]{geometry}
\usepackage[onehalfspacing]{setspace}
\usepackage{graphicx} % zum Einbinden von Graphiken
\usepackage[breaklinks=true,colorlinks=true,linkcolor=blue,urlcolor=blue,citecolor=blue]{hyperref} % f. Referenzen
\usepackage{amsmath,amsthm,amssymb} % Mathematik Umgebung 
\usepackage{icomma} % Intelligentes Komma, das den richtigen Abstand zwischen Dezimalzahlen als auch in Formeln wählt.
\usepackage[ngerman]{babel} % Deutsche Bezeichnungen bei Inhaltsangabe etc
\usepackage[T1]{fontenc}    % andere Schriftsatzkodierung für richtige Silbentrennung bei Umlauten
\usepackage[locale = DE,space-before-unit=true,per-mode = symbol]{siunitx} % Bessere Einheiten
\usepackage{booktabs,multirow} % Pakete zur Erstellung von Tabellen
\usepackage{placeins} % Definiert den Befehl “\FloatBarrier”, der die Ausgabe der davor eingebundenen Bilder erzwingt, befor der Text weiter geht. (Mit vorsicht zu verwenden)
\usepackage[natbib,abbreviate=true,doi=false,style=numeric-comp,giveninits=true,sorting=none]{biblatex} % Modernes Paket zur Erzeugung von Bibliografien (benötigt biber!)
\usepackage{csquotes} % Fortgeschrittene Funktionen für Zitate, für die deutsche Form der Anführungszeichen bei Referenzen
\usepackage{fancyhdr}
\usepackage{enumitem}
\usepackage{float} 
\addbibresource{MyBibliography.bib} % Ort der .bib Datei, die die Datenbank für Literatur/Referenzen enthält.

\graphicspath{{Bilder/}}

\DeclareSIUnit{\dBm}{dBm}
\DeclareSIUnit[per-mode=reciprocal]\WN{\per\centi\meter}

%header:
\pagestyle{fancy}
\fancyhf{}
\fancyhfoffset[L]{1cm} % left extra length
\fancyhfoffset[R]{1cm} % right extra length
\rhead{\today}
\lhead{Numerische Integration und Normalverteilung}
%%%%%%%%%%%%%%%%%%%%%%%%%%%%%%%%%%%%%%%%%%%%%%%%%%%%%%%%%%%%%%%%%%%
\begin{document}
%
\titlehead{\includegraphics[width=5cm]{dhbw.pdf}}
\title{Numerische Integration und Normalverteilung}
\author{Carl-Ferdinand Oliver\thanks{\href{mailto:es23035@lehre.dhbw-stuttgart.de}{es23035@lehre.dhbw-stuttgart.de}}, Tom Springer \thanks{\href{mailto:es23026@lehre.dhbw-stuttgart.de}{es23026@lehre.dhbw-stuttgart.de}}, Jonas Münz \thanks{\href{mailto:es23039@lehre.dhbw-stuttgart.de}{es23039@lehre.dhbw-stuttgart.de}}, \\ Max Valentin Orlemann \thanks{\href{mailto:es23023@lehre.dhbw-stuttgart.de}{es23023@lehre.dhbw-stuttgart.de}}, Alexander Kögel \thanks{\href{mailto:es23031@lehre.dhbw-stuttgart.de}{es23031@lehre.dhbw-stuttgart.de}}\\TES23}
\date{\today}
\maketitle
\vfill

\thispagestyle{empty}
%
%
\tableofcontents
\thispagestyle{empty}
\cleardoublepage
\pagenumbering{arabic} 
\newpage
%
%

%%%%%%%%%%%%%%%%%%%%%%%%%%%%%%%%%%%%%%%%
\section{Statistische Grundlagen}
\label{sec:statistische_grundlagen}

\subsection{Normalverteilung}
\label{sec:normal_distribution}

\subsection{Standardnormalverteilung}
\label{sec:standard_normal_distribution}

\subsection{t-Student-Verteilung}
\label{sec:t_student_distribution}

\subsection{Chi-Quadrat-Verteilung}
\label{sec:chi_square_distribution}


%%%%%%%%%%%%%%%%%%%%%%%%%%%%%%%%%%%%%%%%
\section{Trapezregel}
\label{sec:trapezregel}

\subsection{Herleitung}
\label{sec:herleitung}

\subsection{Fehlerabschätzung}
\label{sec:fehlerabschätzung}

\subsection{Vertrauensniveau $\gamma$}
\label{sec:vertrauensniveau}


%%%%%%%%%%%%%%%%%%%%%%%%%%%%%%%%%%%%%%%%

\section{Umsetzung im Programm}
\label{sec:umsetzung_im_programm}

\subsection{Programmablauf}
\label{sec:programmablauf}

\subsection{Konfidenzintervall}
\label{sec:konfidenzintervall}


%
\printbibliography[]
\vfill


\end{document}
