% Header: Here are all packages used and some additional definitions
%%%%%%%%%%%%%%%%%%%%%%%%%%%%%%%%%%%%%%%%%%%%%%%%%%%%%%%%%%%%%%%%%%%

\documentclass[11pt,a4paper]{scrartcl}
\usepackage[margin=2.5cm]{geometry}
\usepackage[onehalfspacing]{setspace}
\usepackage{graphicx} % zum Einbinden von Graphiken
\usepackage[breaklinks=true,colorlinks=true,linkcolor=blue,urlcolor=blue,citecolor=blue]{hyperref} % f. Referenzen
\usepackage{amsmath,amsthm,amssymb} % Mathematik Umgebung 
\usepackage{icomma} % Intelligentes Komma, das den richtigen Abstand zwischen Dezimalzahlen als auch in Formeln wählt.
\usepackage[ngerman]{babel} % Deutsche Bezeichnungen bei Inhaltsangabe etc
\usepackage[T1]{fontenc}    % andere Schriftsatzkodierung für richtige Silbentrennung bei Umlauten
\usepackage[locale = DE,space-before-unit=true,per-mode = symbol]{siunitx} % Bessere Einheiten
\usepackage{booktabs,multirow} % Pakete zur Erstellung von Tabellen
\usepackage{placeins} % Definiert den Befehl “\FloatBarrier”, der die Ausgabe der davor eingebundenen Bilder erzwingt, befor der Text weiter geht. (Mit vorsicht zu verwenden)
\usepackage[natbib,abbreviate=true,doi=false,style=numeric-comp,giveninits=true,sorting=none]{biblatex} % Modernes Paket zur Erzeugung von Bibliografien (benötigt biber!)
\usepackage{csquotes} % Fortgeschrittene Funktionen für Zitate, für die deutsche Form der Anführungszeichen bei Referenzen
\usepackage{fancyhdr}
\usepackage{enumitem}
\usepackage{float} 
\usepackage{graphicx}
\usepackage{wrapfig}
\addbibresource{MyBibliography.bib} % Ort der .bib Datei, die die Datenbank für Literatur/Referenzen enthält.

\graphicspath{{Bilder/}}

\DeclareSIUnit{\dBm}{dBm}
\DeclareSIUnit[per-mode=reciprocal]\WN{\per\centi\meter}

%header:
\pagestyle{fancy}
\fancyhf{}
\fancyhfoffset[L]{1cm} % left extra length
\fancyhfoffset[R]{1cm} % right extra length
\rhead{\today}
\lhead{Numerische Integration und Normalverteilung}
\fancyfoot[C]{\thepage}
%%%%%%%%%%%%%%%%%%%%%%%%%%%%%%%%%%%%%%%%%%%%%%%%%%%%%%%%%%%%%%%%%%%
\begin{document}
%
\titlehead{\includegraphics[width=5cm]{dhbw.pdf}}
\title{Numerische Integration und Normalverteilung}
\author{Carl-Ferdinand Oliver\thanks{\href{mailto:es23035@lehre.dhbw-stuttgart.de}{es23035@lehre.dhbw-stuttgart.de}}, Tom Springer \thanks{\href{mailto:es23026@lehre.dhbw-stuttgart.de}{es23026@lehre.dhbw-stuttgart.de}}, Jonas Münz \thanks{\href{mailto:es23039@lehre.dhbw-stuttgart.de}{es23039@lehre.dhbw-stuttgart.de}}, \\ Max Valentin Orlemann \thanks{\href{mailto:es23023@lehre.dhbw-stuttgart.de}{es23023@lehre.dhbw-stuttgart.de}}, Alexander Kögel \thanks{\href{mailto:es23031@lehre.dhbw-stuttgart.de}{es23031@lehre.dhbw-stuttgart.de}}\\TES23}
\date{\today}
\maketitle
\vfill

\thispagestyle{empty}
%
%
\tableofcontents
\thispagestyle{empty}
\cleardoublepage
\pagenumbering{arabic} 
\newpage
%
%

%%%%%%%%%%%%%%%%%%%%%%%%%%%%%%%%%%%%%%%%
%Statistische Grundlagen
\section{Statistische Grundlagen}
\label{sec:statistische_grundlagen}

\subsection{Normalverteilung}
\label{sec:normal_distribution}
Die Gaußsche Normalverteilung beschreibt die Verteilung einer stetigen Zufallsvariablen und ist durch zwei Parameter, den Mittelwert $\mu$ und die Standardabweichung $\sigma$, charakterisiert. Sie wird auch als \textit{Glockenkurve} bezeichnet und ist symmetrisch um den Mittelwert.

\subsection*{Wichtige Eigenschaften}
\begin{itemize}
    \item Die Dichtefunktion der Normalverteilung lautet:
    \[
    f(x) = \frac{1}{\sqrt{2 \pi} \sigma} \cdot \exp\left(-\frac{(x - \mu)^2}{2 \sigma^2}\right)
    \]
    \item Die Verteilungsfunktion ist gegeben durch:
    \[
    \Phi(x) = P(X \leq x) = \int_{-\infty}^{x} f(t) \, dt
    \]
    \item Die Normalverteilung ist symmetrisch zum Mittelwert $\mu$, hat ein Maximum bei $x = \mu$ und Wendepunkte bei $x = \mu \pm \sigma$.
    \item Sie ist normiert, d.h., das Integral über die gesamte Dichtefunktion ergibt 1:
    \[
    \int_{-\infty}^{\infty} f(x) \, dx = 1
    \]
\end{itemize}

\subsection*{Berechnung der Wahrscheinlichkeiten}
Für die Berechnung von Wahrscheinlichkeiten, die der Gaußschen Normalverteilung folgen, werden in der Praxis oft Tabellen oder numerische Verfahren verwendet, da die Integrale analytisch nicht lösbar sind. Die Wahrscheinlichkeit für einen Bereich $a \leq X \leq b$ kann mit der Verteilungsfunktion berechnet werden:
\[
P(a \leq X \leq b) = \Phi\left(\frac{b - \mu}{\sigma}\right) - \Phi\left(\frac{a - \mu}{\sigma}\right)
\]

\subsection*{Wahrscheinlichkeiten in Abhängigkeit der Standardabweichung}

Da die Normalverteilung symmetrisch um den Mittelwert $\mu$ ist, lassen sich bestimmte Wahrscheinlichkeitsbereiche in Abhängigkeit von der Standardabweichung $\sigma$ um $\mu$ angeben:

\begin{itemize}
    \item Etwa 68,27\% aller Werte einer normalverteilten Zufallsvariablen liegen im Intervall $[\mu - \sigma, \mu + \sigma]$, also innerhalb einer Standardabweichung um den Mittelwert.
    \item Etwa 95,45\% der Werte befinden sich im Intervall $[\mu - 2\sigma, \mu + 2\sigma]$, also innerhalb von zwei Standardabweichungen.
    \item Rund 99,73\% der Werte liegen im Intervall $[\mu - 3\sigma, \mu + 3\sigma]$, also innerhalb von drei Standardabweichungen.
\end{itemize}

Diese Bereiche werden auch als \textit{Empirische Regel} oder \textit{68-95-99,7-Regel} bezeichnet und sind besonders nützlich zur Einschätzung, wie wahrscheinlich es ist, dass eine Zufallsvariable in einem bestimmten Bereich um den Mittelwert liegt. Die Wahrscheinlichkeit, dass ein Wert weiter als $3\sigma$ vom Mittelwert entfernt liegt, ist sehr gering und beträgt nur ca. 0,27\%.


\subsection{Standardnormalverteilung}
\label{sec:standard_normal_distribution}

Eine Normalverteilung lässt sich durch die Standardisierung mit $z = \frac{x - \mu}{\sigma}$ auf die sogenannte Standardnormalverteilung $\mathcal{N}(0, 1)$ zurückführen, die Mittelwert 0 und Standardabweichung 1 hat:
\[
\phi(z) = \frac{1}{\sqrt{2 \pi}} \cdot \exp\left(-\frac{z^2}{2}\right)
\]

\subsection{t-Student-Verteilung}
\label{sec:t_student_distribution}

\subsection{Chi-Quadrat-Verteilung}
\label{sec:chi_square_distribution}




%%%%%%%%%%%%%%%%%%%%%%%%%%%%%%%%%%%%%%%%
%Statistische Grundlagen
\section{Trapezregel}
\label{sec:trapezregel}
Diese wird zur numerischen Integration gebraucht, beipielweise um Warscheinlichkeit in einer Normalverteilung zu berechnen. Diese Integrale können, wie in \autoref{sec:statistische_grundlagen} besprochen, nicht analytisch gelöst werden und müssen damit genähert werden. Die Trapezregel ist eine einfache Methode hierzu.
\subsection{Herleitung}
\label{sec:herleitung}

\begin{figure}[h]
    \centering
    \includegraphics[width=8cm]{Bilder/keplersche_fassregel_funktion.png}
    \caption{Zu integrierende Funktion \cite{skript}}    
    \label{fig:keplersche_fassregel}
\end{figure}
Es wird ein Funktion $f$ betrachtet, deren Schaubild im gewünschten Intervall $I = [a, b]$ in \autoref{fig:sehnentrapeze_eins} gezeigt ist.

\begin{figure}[h]
  \centering
        \includegraphics[width=8cm]{Bilder/sehnentrapeze.png}
      \caption{Sehnentrapeze für die anstehende Rechnung}
      \label{fig:sehnentrapeze_eins}
    \end{figure}
\subsection*{Zwei Sehnentrapeze}

Zur Flächenberechnung werden zwei Sehnentrapeze in das gegebene Schaubild gezeichnet, wie in \autoref{fig:sehnentrapeze_eins} zu sehen ist. Für die Fläche eines Trapezes mit den Grundseiten $a$ und $b$ und der Höhe $h$ gilt:
\[
A_{\text{Trapez}} = \frac{a + b}{2} \cdot h = m \cdot h
\]
Übertragen auf die beiden Sehnentrapeze aus \autoref{fig:sehnentrapeze_eins} ergibt sich die Gesamtfläche $S$ als:
\[
S = \frac{b - a}{2} \cdot \left( \frac{f(a) + f\left( \frac{a + b}{2} \right)}{2} + \frac{f(b) + f\left( \frac{a + b}{2} \right)}{2} \right)
\]
\[
  \Leftrightarrow S = \frac{b - a}{2} \cdot \left(\frac{f(a)}{2} + f\left(\frac{a + b}{2}\right) + \frac{f(b)}{2}\right)
\]

\subsection*{Die summierte Trapezregel}
Nun wird versucht, die Funktion über mehr als zwei Trapeze anzunähern. Dazu unterteilen wir das Intervall $I = [a, b]$ in $n$ gleich große Teilintervalle. Die Funktion \( f(x) \) wird folgendermaßen interpoliert:
\[
I = \int_a^b f(x) \, dx \approx \frac{b-a}{2n} \sum_{i=0}^{n-1} \left(f(x_i) + f(x_{i+1})\right),
\]
\[
  \Leftrightarrow I = \frac{b - a}{n} \cdot \left( \frac{f(a)}{2} + f \left( a + \frac{b - a}{n} \right) + \cdots + f \left( b - \frac{b - a}{n} \right) + \frac{f(b)}{2} \right)
\]
\[
x_i = a + i \cdot \frac{b-a}{n}, \quad i = 0, 1, 2, \ldots, n-1.
\]

Für den Fall, dass die Anzahl der Intervalle \( n \) eine gerade Zahl ist, gilt die Näherung:
\[
I \approx \frac{b-a}{2} \left[ \frac{f(a)}{2} - f \left(a + \frac{b-a}{n}\right) + \ldots - f \left(b - \frac{b-a}{n}\right) - \frac{f(b)}{2} \right].
\]
Die summierte Trapezregel ist ein Verfahren 2.Ordnung, d.h. der Fehler ist proportional zu der Schrittweite $h = x_{i+1} - x_i$ im Quadrat. \cite{vollskript}


\begin{figure}[h]
  \centering
        \includegraphics[width=8cm]{Bilder/Trapezregel.png}
      \caption{Illustration zur summierten Trapezregel \cite{vollskript}}
      \label{fig:sehnentrapeze_eins}
    \end{figure}
%
%\subsection*{B. Das Tangententrapez}
%
%Nun legen wir ein weiteres Tangententrapez in das Schaubild, wie in \autoref{fig:sehnentrapeze_zwei} dargestellt. Nach der Trapezformel ergibt sich:
%\[
%T = f\left(\frac{a + b}{2}\right) \cdot (b - a)
%\]
%Da wir \textbf{doppelt so viele Sehnentrapeze wie Tangententrapeze} haben, gewichten wir die Flächen entsprechend. Die Gesamtfläche $I[f]$ nähert sich durch:
%\[
%I[f] \approx A = \frac{1}{3} \cdot (2S + T)
%\]
%\[
%  \Leftrightarrow A = \frac{b - a}{6} \cdot \left(f(a) + 4 \cdot f\left(\frac{a + b}{2}\right) + f(b)\right)
%\]
%
%\subsection*{Die Keplersche Fassregel}
%
%Ist die Funktion $f$ auf dem Intervall $I = [a, b]$ stetig, so gilt die Keplersche Fassregel:
%\[
%I[f] \approx A = \frac{b - a}{6} \cdot \left(f(a) + 4 \cdot f\left(\frac{a + b}{2}\right) + f(b)\right)
%\]
%Dabei ist zu beachten, dass die Keplersche Fassregel nur dann gute Näherungswerte liefert, wenn sich die Funktion im betrachteten Intervall durch eine Parabel annähern lässt (z.B. eine Normalverteilung). Daher ist es ratsam, das Intervall $[a, b]$ in \textbf{viele kleine, gleich große Teilintervalle} zu unterteilen und die Keplersche Fassregel auf jedes Teilintervall anzuwenden. Dabei sollte für eine bessere Genauigkeit das Intervall in möglichst viele, kleine Intervalle unterteilt werden. \cite{skript}
%
\subsection{Fehlerabschätzung}
\label{sec:fehlerabschätzung}

Für die Fehlerberechnung muss zunächst überhaupt definiert werden, was dieser Fehler ist:
\subsection*{Definition des Fehlers}
Der Fehler wird definiert als die Breite des Konfidenzintervalls:
\[
\text{Breite} = 2 \cdot c \cdot \frac{\sigma}{\sqrt{n}}
\]
Hierbei ist $c$ eine Konstante und wird mit der Trapezregel berechnet.

\subsection*{Berechnung des Standardfehlers}
Der Standardfehler (SE) des Mittelwerts wird berechnet als:
\[
SE = \frac{\sigma}{\sqrt{n}}
\]
Daraus folgt die Fehlerabschätzung:
\[
\text{Fehlerabschätzung} = c \cdot SE
\]
\subsection*{Interpretation}
Ein kleiner Fehler zeigt an, dass das Konfidenzintervall eine präzise Schätzung des wahren Mittelwerts bietet. Ein größerer Fehler deutet auf mögliche Unsicherheiten in den Daten oder eine unzureichende Stichprobengröße hin. Beispielsweise können so auch die Genauigkeit von Sensordaten oder ähnlichen eingeschätzt werden. 


\subsection{Vertrauensniveau $\gamma$}
\label{sec:vertrauensniveau}
Das Vertrauensniveau \(\gamma\) ist ein Wert, mit dem Angegeben wird wie sehr den Messwerte den wahren Wert des Parameters enthält (sprich wie genau die Sensoren sind). Es wird in der Regel als Dezimalzahl zwischen 0 und 1 dargestellt, wobei typische Werte 0.95 (95\%) und 0.99 (99\%) sind.

\begin{itemize}
    \item \textbf{Einfluss auf die Breite des Intervalls:} Ein höheres Vertrauensniveau führt zu einem breiteren Konfidenzintervall. Das bedeutet, dass wir mit größerer Sicherheit sagen können, dass der wahre Wert in diesem Intervall liegt. Ein niedrigeres Vertrauensniveau hingegen führt zu einem schmaleren Intervall, was die Schätzung präziser macht, aber auch das Risiko erhöht, dass der wahre Wert außerhalb liegt.
    
    \item \textbf{Berechnung:} Das Vertrauensniveau wird verwendet, um kritische Werte zu bestimmen, die notwendig sind, um das Konfidenzintervall zu berechnen. 
\end{itemize}



%%%%%%%%%%%%%%%%%%%%%%%%%%%%%%%%%%%%%%%%

%Statistische Grundlagen

\section{Umsetzung im Programm}
\label{sec:umsetzung_im_programm}

\subsection{Berechnung des Mittelwerts}
Der Mittelwert (\(\mu\)) der Daten wird berechnet, indem die Summe aller Datenpunkte durch die Anzahl der Punkte (\(n\)) geteilt wird:
\[
\mu = \frac{1}{n} \sum_{i=1}^{n} A_i
\]

\subsection{Berechnung der Varianz}
Die Varianz (\(\sigma^2\)) wird berechnet, um zu messen, wie die Daten um den Mittelwert streuen. Der Code verwendet die Formel für die erwartungstreue Varianz:
\[
\sigma^2 = \frac{1}{n-1} \sum_{i=1}^{n} (A_i - \mu)^2
\]
Die Verwendung von \(n-1\) anstelle von \(n\) ist wichtig, um eine Verzerrung der Schätzung zu vermeiden (Bessel-Korrektur).

\subsection{Berechnung der Konstante \(c\) durch numerische Integration}
Der Code verwendet die Trapezregel, um die Fläche unter der Standardnormalverteilung zu berechnen. Die Trapezregel ist eine numerische Methode zur Approximation des Integrals einer Funktion.

\subsubsection{Ziel der Berechnung}
Die Fläche unter der Standardnormalverteilung wird so lange berechnet, bis sie einen bestimmten Zielwert erreicht, der durch \((\gamma + 1) / 2\) definiert ist. Dies entspricht dem gewünschten Konfidenzniveau (z. B. 99\% oder 0,99).

\subsubsection{Funktion der Standardnormalverteilung}
Die Funktion \texttt{standardnormalverteilung(x)} berechnet den Wert der Standardnormalverteilung für einen gegebenen Wert \(x\):
\[
f(x) = \frac{1}{\sqrt{2\pi}} e^{-\frac{x^2}{2}}
\]

\subsection{Berechnung des Konfidenzintervalls}
Nach der Berechnung der Konstante \(c\) wird das Konfidenzintervall mit der folgenden Formel berechnet:
\[
\text{Oben} = \mu + c \cdot \sqrt{\frac{\sigma^2}{n}}
\]
\[
\text{Unten} = \mu - c \cdot \sqrt{\frac{\sigma^2}{n}}
\]
Dies gibt das Konfidenzintervall an, in dem mit einer bestimmten Wahrscheinlichkeit (z.B. 99\%) der wahre Mittelwert der Grundgesamtheit liegt.


%
\printbibliography[]
\vfill


\end{document}
