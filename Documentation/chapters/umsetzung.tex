\section{Umsetzung im Programm}
\label{sec:umsetzung_im_programm}
Die Umsetzung des Programms erfolgt in Python mithilfe von numpy. Die Berechnung des Mittelwertes, der Grenzen des Konfidenzintervalls, der Varianz $\sigma^2$ und der Konstante $c$ erfolgt in der Funktion \texttt{konfidenz(dateiname, gamma)}.
\subsection{Berechnung des Mittelwerts}
Der Mittelwert (\(\mu\)) der Daten wird berechnet, indem die Summe aller Datenpunkte durch die Anzahl der Punkte (\(n\)) geteilt wird:
\[
\mu = \frac{1}{n} \sum_{i=1}^{n} A_i
\]

\subsection{Berechnung der Varianz}
Die Varianz (\(\sigma^2\)) wird berechnet, um zu messen, wie die Daten um den Mittelwert streuen. Der Code verwendet folgende Formel für die erwartungstreue Varianz:
\[
\sigma^2 = \frac{1}{n-1} \sum_{i=1}^{n} (A_i - \mu)^2
\]

\subsection{Berechnung der Konstante \(c\) durch numerische Integration}
Der Code verwendet die Trapezregel, um die Fläche unter der Standardnormalverteilung zu berechnen. Die Formel wird in \autoref{sec:herleitung} hergeleitet.

\subsubsection{Ziel der Berechnung}
Die Fläche unter der Standardnormalverteilung wird so lange berechnet, bis sie einen bestimmten Zielwert erreicht, der durch \((\gamma + 1) / 2\) definiert ist. Dies entspricht dem gewünschten Konfidenzniveau (hier gewählt als 99\% $\vee$ 0,99).

\subsubsection{Funktion der Standardnormalverteilung}
Die Funktion \texttt{standardnormalverteilung(x)} berechnet den Wert der Standardnormalverteilung für einen gegebenen Wert \(x\):
\[
f(x) = \frac{1}{\sqrt{2\pi}} e^{-\frac{x^2}{2}}
\]

\subsection{Berechnung des Konfidenzintervalls}
Nach der Berechnung der Konstante \(c\) wird das Konfidenzintervall mit der folgenden Formel berechnet:
\[
\text{Oben} = \mu + c \cdot \sqrt{\frac{\sigma^2}{n}}
\]
\[
\text{Unten} = \mu - c \cdot \sqrt{\frac{\sigma^2}{n}}
\]
Damit können wir die beiden Grenzen des Konfidenzintervalls herausfinden, zwischen denen mit einer bestimmten Wahrscheinlichkeit (von uns gewählt als 99\%) der wahre Mittelwert der Grundgesamtheit liegt.

